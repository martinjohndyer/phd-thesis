\pdfbookmark[section]{Summary}{summary}
\chapter*{Summary}

\begin{onehalfspace}

The detection of the first electromagnetic counterpart to a gravitational-wave signal in August 2017 marked the start of a new era of multi-messenger astrophysics. An unprecedented number of telescopes around the world were involved in hunting for the source of the signal, and although more gravitational-wave signals have been since detected, no further electromagnetic counterparts have been found.

\medskip

In this thesis, I present my work to help build a telescope dedicated to the hunt for these elusive sources: the Gravitational-wave Optical Transient Observer (GOTO). I detail the creation of the GOTO Telescope Control System, G-TeCS, which includes the software required to control multiple wide-field telescopes on a single robotic mount. G-TeCS also includes software that enables the telescope to complete a sky survey and transient alert follow-up observations completely autonomously, whilst monitoring the weather conditions and automatically fixing any hardware issues that arise. I go on to describe the routines used to determine target priorities, as well as how the all-sky survey grid is defined, how gravitational-wave and other transient alerts are received and processed, and how the optimum follow-up strategies for these events were determined.

\medskip

The first GOTO telescope, situated on La Palma in the Canary Islands, saw first light in June 2017. I detail the work I carried out on the site to help commission the prototype, and how the control software was developed during the commissioning phase. I also analyse the GOTO CCD cameras and optics, building a complete theoretical model of the system to confirm the performance of the prototype. Finally, I describe the results of simulations I carried out of the future of the GOTO project, with multiple robotic telescopes on La Palma and in Australia, and how the G-TeCS software might be modified to operate these telescopes as a single, global observatory.

\end{onehalfspace}
