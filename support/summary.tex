\pdfbookmark[section]{Summary}{summary}
\chapter*{Summary}

\begin{onehalfspace}

The detection of the first electromagnetic counterpart to a gravitational-wave signal, in August 2017, marked the start of a new era of multi-messenger astrophysics. The GW170817 gravitational-wave detection, the first from a binary neutron star, was followed by an unprecedented international campaign to locate the counterpart source. This effort was helped by the source being localised to within a relatively-small 30 square degree region on the sky, and the counterpart was eventually found in otherwise unremarkable galaxy approximately 140 million light-years away. The second detection of gravitational waves from a binary neutron star, in April 2019, could only be localised to within an area on the sky covering 8,000 square degrees, making finding the counterpart source an almost impossible task. No counterpart source to this second detection was discovered, despite another major search effort.

This thesis presents work my work to help build a telescope dedicated to the hunt for these elusive sources: the Gravitational-wave Optical Transient Observer (GOTO). The GOTO project plans to use multiple wide-field robotic telescopes to automatically search for optical counterparts whenever gravitational-wave signals are detected. The first GOTO prototype telescope was commissioned on La Palma in the Canary Islands in July 2017, and the project is ultimately planned to include multiple networked telescopes at sites across the globe.

My work on the GOTO project has focused on building the software and infrastructure to control these telescopes. In this thesis I detail the creation of a custom telescope control system, G-TeCS, which encompasses the software required to process gravitational-wave alerts, schedule observations and control the GOTO telescope. I also analyse the GOTO cameras and optical hardware, detail work carried out at the site on La Palma to commission the GOTO prototype, and finish by describing simulations of the future global GOTO observatory.

\end{onehalfspace}
