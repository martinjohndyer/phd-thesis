\chapter{Conclusions and Future Work}
\label{chap:conclusion}
\chaptoc{}

% ########################################

\newpage
\section{Summary and Conclusions}
\label{sec:conclusion}
\begin{colsection}

% ~~~~~~~~~~~~~~~~~~~~

\begin{colsection}

In this thesis I have described my work as part of the GOTO project, primarily working on the control software in order to create a fully-autonomous robotic telescope. After several years of development and commissioning the prototype GOTO telescope is fully operational, and observing from La Palma most nights with no human interaction.

\end{colsection}

% ~~~~~~~~~~~~~~~~~~~~

\subsection{Telescope control}
\label{sec:control_results}
\begin{colsection}

The core of my work has been the \glsfirst{gtecs}, a complete software package split across several Python modules that controls every aspect of the telescope. The hardware control daemons interface with the dome, mount and cameras (see \aref{chap:gtecs}) while the ``pilot'' master control program and its associated systems (see \aref{chap:autonomous}) allow the telescope to function with no human involvement. GOTO has now been operating successfully for months with the pilot in full control. The conditions monitoring systems have proven robust enough to trust the telescope to close in bad weather, and when the occasional unexpected hardware issues do occur the pilot recovery systems can fix the problem and resume observing in a majority of cases before a human even has time to log in. Of course commissioning was not entirely without incident, as described in \aref{chap:commissioning}. However all the software challenges were overcome, and the majority of the delays to GOTO were due to hardware faults which were out of my purview.

Each set of exposures taken with the G-TeCS camera daemon are assigned an incremental run number. From the initial installation in the summer of 2017 up until September 2019 GOTO has taken over 185,000 such exposure sets, and produced many tens of terabytes of data. My role ends when each image has been saved to a file, whereupon they are copied to the master archive in Warwick and backups at other partner institutions.

\end{colsection}

% ~~~~~~~~~~~~~~~~~~~~
\newpage
\subsection{Scheduling and alert follow-up}
\label{sec:gw_results}
\begin{colsection}

GOTO needed a observation scheduling system that could deal with both the survey and follow-up modes. The scheduler used by G-TeCS is a just-in-time system, where the highest priority target is recalculated every time the scheduler is called. This makes it very reactive to transient alerts, which was a requirement of the project. As described above the scheduler has been operating as parted of the automated systems for several months, and GOTO has been reliably taking regular observations.

From the beginning of the LIGO-Virgo third observing run (O3) in April 2019 up until the end of August the LVC released 32 alerts. The G-TeCS sentinel received and reacted to every one of these events, with the results given in \aref{tab:obs_log}. In a few cases the event handler initially failed to process the VOEvent or the skymap, and required manual intervention. As described in
\aref{sec:challenges}
this was usually due to a problem in the LIGO GraceDB, and changes to the GOTO-alert code were made to work around the problem in the future. Each event had pointings added to the observation database, and observations were taken for 25 of the 32 events, of which three were later retracted. Of the remaining seven cases four alerts were received during the day on La Palma and were then retracted before sunset. Only three of the 25 real events had no part of the skymap visible from La Palma.

For events that occurred during the night on La Palma the GOTO-alert event handling system allowed the pilot to immediately begin observations of the visible skymap. As shown in \aref{tab:obs_log} this was the case for eight of the 32 alerts, and in all but one case the first exposure was started less than 60 seconds after the sentinel received the notice. The time delay varies between 30 and 56 seconds, this range will depend on how far the mount had to slew from its previous target. A significant amount of the remaining delay due to having to download the large LVC skymaps from GraceDB, with the rest due to various small delays in the sentinel and pilot, such as the pilot needing to wait for the next scheduler check. Future optimisation might be possible to reduce these delays further. The one exception to this pattern was event S190513bm, which was immediately visible but observations were delayed by 4 minutes. In this case the observations could have started earlier, but at the time the alert was received the pilot was already observing a pointing from the S190512at event received the previous day. As both events were black hole binaries they were inserted at the same rank, and as detailed in
\aref{sec:event_strategy}
equal-rank ToO pointings won't interrupt each other, so the new pointing had to wait until the previous one was completed. In all other cases the pilot was observing a lower-rank target, usually a survey tile, which was immediately aborted when the scheduler check returned the ToO gravitational wave pointing.

\begin{sidewaystable}[p]
    \begin{footnotesize}
    \begin{center}
        \begin{tabular}{l|cccrl} % chktex 44

            \multicolumn{1}{c|}{Event} &
            Event time &
            Alert received &
            Observation start &
            \multicolumn{1}{c}{$\Delta T$ (h)} &
            Notes
            \\
            \midrule
            \textcolor{Red}{S190405ar} &
            2019--04--05 16:01:30 &
            2019--04--12 15:07:26 &
            --- &
            --- &
            \textit{(Retracted before sunset)}
            \\
            S190408an &
            2019--04--08 18:18:02 &
            2019--04--08 19:02:50 &
            2019--04--09 05:40:39 &
            10.63 &

            \\
            S190412m &
            2019--04--12 05:30:44 &
            2019--04--12 06:31:39 &
            2019--04--12 20:28:35 &
            13.95 &
            See GCN \citet{GW190412_GOTO}
            \\
            S190421ar &
            2019--04--21 21:38:56 &
            2019--04--22 16:26:24 &
            2019--04--23 21:54:59 &
            29.48 &

            \\
            S190425z &
            2019--04--25 08:18:05 &
            2019--04--25 09:00:56 &
            2019--04--25 20:38:22 &
            11.62 &
            See GCN \citet{GW190425_GOTO}
            \\
            S190426c &
            2019--04--26 15:21:55 &
            2019--04--26 15:47:11 &
            2019--04--26 20:38:45 &
            4.86 &
            See GCN \citet{GW190426_GOTO}
            \\
            S190503bf &
            2019--05--03 18:54:04 &
            2019--05--03 19:30:15 &
            --- &
            --- &
            \textit{(Never visible from La Palma)}
            \\
            S190510g &
            2019--05--10 02:59:39 &
            2019--05--10 04:21:59 &
            2019--05--10 04:22:55 &
            0.02 &
            Observations began \textbf{56\,s} after notice received
            \\
            S190512at &
            2019--05--12 18:07:14 &
            2019--05--12 18:59:01 &
            2019--05--12 20:53:20 &
            1.91 &

            \\
            S190513bm &
            2019--05--13 20:54:28 &
            2019--05--13 21:21:51 &
            2019--05--13 21:26:19 &
            0.07 &
            Observations began \textbf{4\,min} after notice received
            \\
            S190517h &
            2019--05--17 05:51:01 &
            2019--05--17 06:26:48 &
            2019--05--17 21:42:06 &
            15.26 &

            \\
            \textcolor{Red}{S190518bb} &
            2019--05--18 19:19:19 &
            2019--05--18 19:25:49 &
            --- &
            --- &
            \textit{(Retracted before sunset)}
            \\
            S190519bj &
            2019--05--19 15:35:44 &
            2019--05--19 17:01:40 &
            2019--05--19 20:55:19 &
            3.89 &

            \\
            S190521g &
            2019--05--21 03:02:29 &
            2019--05--21 03:08:49 &
            2019--05--21 03:09:17 &
            0.01 &
            Observations began \textbf{28\,s} after notice received
            \\
            S190521r &
            2019--05--21 07:43:59 &
            2019--05--21 07:50:27 &
            2019--05--21 22:54:03 &
            15.06 &

            \\
            \textcolor{Red}{S190524q} &
            2019--05--24 04:52:06 &
            2019--05--24 04:58:40 &
            2019--05--24 04:59:33 &
            0.01 &
            Observations began \textbf{53\,s} after notice received
            \\
            S190602aq &
            2019--06--02 17:59:27 &
            2019--06--02 18:06:01 &
            --- &
            --- &
            \textit{(Never visible from La Palma)}
            \\
            S190630ag &
            2019--06--30 18:52:05 &
            2019--06--30 18:55:47 &
            2019--06--30 21:14:49 &
            2.32 &

            \\
            S190701ah &
            2019--07--01 20:33:06 &
            2019--07--01 20:38:06 &
            --- &
            --- &
            \textit{(Never visible from La Palma)}
            \\
            S190706ai &
            2019--07--06 22:26:41 &
            2019--07--06 22:44:31 &
            2019--07--06 22:45:09 &
            0.01 &
            Observations began \textbf{38\,s} after notice received
            \\
            S190707q &
            2019--07--07 09:33:26 &
            2019--07--07 10:13:24 &
            2019--07--07 21:54:47 &
            11.69 &

            \\
            S190718y &
            2019--07--18 14:35:12 &
            2019--07--18 15:03:13 &
            2019--07--18 21:08:53 &
            6.09 &

            \\
            S190720a &
            2019--07--20 00:08:36 &
            2019--07--20 00:11:26 &
            2019--07--20 00:11:57 &
            0.01 &
            Observations began \textbf{31\,s} after notice received
            \\
            S190727h &
            2019--07--27 06:03:33 &
            2019--07--27 06:12:02 &
            2019--07--27 21:03:40 &
            14.86 &

            \\
            S190728q &
            2019--07--28 06:45:10 &
            2019--07--28 06:59:32 &
            2019--07--28 21:29:58 &
            14.51 &

            \\
            \textcolor{Red}{S190808ae} &
            2019--08--08 22:21:21 &
            2019--08--08 22:28:00 &
            2019--08--08 22:28:31 &
            0.01 &
            Observations began \textbf{31\,s} after notice received
            \\
            S190814bv &
            2019--08--14 21:10:39 &
            2019--08--14 21:31:44 &
            2019--08--14 22:59:27 &
            1.46 &
            See GCN \citet{GW190814_GOTO}
            \\
            \textcolor{Red}{S190816i} &
            2019--08--16 13:04:31 &
            2019--08--16 13:11:35 &
            --- &
            --- &
            \textit{(Retracted before sunset)}
            \\
            \textcolor{Red}{S190822c} &
            2019--08--22 01:29:59 &
            2019--08--22 01:37:00 &
            2019--08--22 01:37:30 &
            0.01 &
            Observations began \textbf{30\,s} after notice received
            \\
            S190828j &
            2019--08--28 06:34:05 &
            2019--08--28 06:50:14 &
            2019--08--28 22:38:25 &
            15.80 &

            \\
            S190828l &
            2019--08--28 06:55:09 &
            2019--08--28 07:17:46 &
            2019--08--28 23:48:38 &
            16.51 &

            \\
            \textcolor{Red}{S190829u} &
            2019--08--29 21:05:56 &
            2019--08--29 21:17:14 &
            --- &
            --- &
            \textit{(Retracted before sunset)}
            \\

        \end{tabular}
    \end{center}
    \end{footnotesize}
    \caption[GOTO observation log for O3 events so far]{
        GOTO observation log for O3 events up to the end of August 2019. \textcolorbf{Red}{Red} events were later retracted.
    }\label{tab:obs_log}
\end{sidewaystable}

As described back in \aref{sec:gw_detections}, three of the 25 real events detected in O3 so far were the most likely to have a potential electromagnetic counterpart: S190425z, S190426c and S190814bv were all classified as originating from either a binary neutron star or neutron star-black hole binary. These were the most important events to follow up, and the GOTO response to each was reported in the GCN notices \citet{GW190425_GOTO, GW190426_GOTO, GW190814_GOTO}. After analysis no counterparts were found for either of the three events, by GOTO or other projects. The S190814bv skymap had a 90\% contour covering only 23 square degrees, which made it very easy to cover with GOTO;\@ on the other hand the initial skymaps for S190425z and S190426c covered areas of 10,000 sq deg and 1,900 sq deg respectively. The follow-up of both was also complicated by having two viable events occurring within 36 hours of each other. Ultimately in the first two days after the alerts GOTO covered 30\% of the S190425z probability \citep{GW190425_GOTO} and 54\% of the S190426c probability \citep{GW190426_GOTO}, and the coverage maps are shown in \aref{fig:190425_goto} and \aref{fig:190426_goto}.

For these events the follow-up code has been shown to be fast and reliable. Based on this performance if, or when, another GW170817-like event is detected GOTO should be ready and observing the counterpart within 60 seconds of the alert being received.

\begin{figure}[p]
    \begin{center}
        \includegraphics[width=0.9\linewidth]{images/190425_goto.pdf}
    \end{center}
    \caption[Follow-up observations of S190425z with GOTO]{
        Follow-up observations of S190425z with GOTO.\@ The tiled observations are shown in \textcolorbf{NavyBlue}{blue}, over the background BAYESTAR skymap in \textcolorbf{Orange}{orange}. Compare to \aref{fig:ztf} showing ZTF's coverage of the same event.
        }\label{fig:190425_goto}
\end{figure}

\begin{figure}[p]
    \begin{center}
        \includegraphics[width=0.9\linewidth]{images/190426_goto.pdf}
    \end{center}
    \caption[Follow-up observations of S190426c with GOTO]{
        Follow-up observations of S190426c with GOTO.\@ The tiled observations are shown in \textcolorbf{NavyBlue}{blue}, over the background BAYESTAR skymap in \textcolorbf{Orange}{orange}.
        }\label{fig:190426_goto}
\end{figure}

\clearpage

\end{colsection}

% ~~~~~~~~~~~~~~~~~~~~

\end{colsection}

% ########################################

\newpage
\section{Future work}
\label{sec:future}
\begin{colsection}

% ~~~~~~~~~~~~~~~~~~~~

\begin{colsection}

\todo{WIP}
\citep{GW150914, GW150914_followup, GW151226}

\end{colsection}

% ~~~~~~~~~~~~~~~~~~~~

\subsection{The global control system}
\label{sec:gtecs_multisite}
\begin{colsection}

\todo{WIP}

\end{colsection}

% ~~~~~~~~~~~~~~~~~~~~

\end{colsection}

% ########################################
