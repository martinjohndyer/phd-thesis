\chapter{Simulating the GOTO System}
\label{chap:sims}
\chaptoc{}

% ########################################

\newpage
\section{Introduction}
\label{sec:sims_intro}
\begin{colsection}

% ~~~~~~~~~~~~~~~~~~~~

\begin{colsection}

In this chapter I outline my work creating simulations of the GOTO observatory. Section~\ref{sec:scheduler_sims} describes simulations carried out in order to find the optimal ``tie-break'' parameters for the \gls{gtecs} scheduler, while Section~\ref{sec:multitel_sims} describes simulations of the combined response of multiple GOTO telescopes observing gravitational wave skymaps and the all-sky survey. All work described in this chapter is my own and has not been published before anywhere else.

\end{colsection}

% ~~~~~~~~~~~~~~~~~~~~

\subsection{Simulating GOTO}
\label{sec:goto_sims}
\begin{colsection}

WIP

\end{colsection}

% ~~~~~~~~~~~~~~~~~~~~

\end{colsection}

% ########################################

\newpage
\section{Scheduler simulations}
\label{sec:scheduler_sims}
\begin{colsection}

% ~~~~~~~~~~~~~~~~~~~~

\begin{colsection}

The scheduler is described in in Section~\ref{sec:scheduler}.

Why do we need a tie-break? (described previously a bit, but worth reminding)

The different tie-break parameters:
\begin{itemize}
    \item Airmass, obviously
    \item Weight, also obviously
    \item Time to set / time valid?
\end{itemize}

Simulations

Output: plot airmass vs time observed

\rtxt{still work to do}

\end{colsection}

% ~~~~~~~~~~~~~~~~~~~~
\end{colsection}

% ########################################

\newpage
\section{Multi-telescope simulations}
\label{sec:multitel_sims}
\begin{colsection}

% ~~~~~~~~~~~~~~~~~~~~

\begin{colsection}

GOTO will have multiple telescopes. Also Australia (see Section~\ref{sec:goto_expansion}).

These simulations were carried out to provide some predictions for what benefit multiple telescopes and sites will provide.


\end{colsection}

% ~~~~~~~~~~~~~~~~~~~~

\subsection{Simulating multiple telescopes and sites}
\label{sec:multi_site}
\begin{colsection}

Multi telescopes are easy, just take top X.

Multi sites are harder. Luckily night times don't overlap.

But what if they did?

They need to be on the same grid.

BUT, can combine different grids at different sites.

\end{colsection}

% ~~~~~~~~~~~~~~~~~~~~

\subsection{Gravitational wave follow-up simulations}
\label{sec:gw_sims}
\begin{colsection}

First 2 years (could be in prev \autoref{sec:scheduler_sims})

Timing and positions are pretty evenly spread (check paper).

Visibility: depends on sites, position of the Sun etc.

Selection is a small issue, some are just so far from the contours. Differs depending on the grid. See \autoref{sec:db_insert}.

How the simulations worked: real GOTO-alert commands, real database, real scheduler, fake pilot.

Results.

Analysis:
\begin{itemize}
    \item Different grids are different.
    \item Small gain from additional telescopes at the same site.
    \item Huge gain from multiple sites.
\end{itemize}

Still to do:
\begin{itemize}
    \item weather
    \item SA?\@ Chile?
\end{itemize}

\end{colsection}

% ~~~~~~~~~~~~~~~~~~~~

\subsection{All-sky survey simulations}
\label{sec:allsky_sims}
\begin{colsection}

Simulate over a whole year.

Using the real scheduler takes a long time.

Light version is much easer, just selects based on altitude.

Get results for cadence, coverage, airmass?

Compare to real results over first 5 months, Feb-Jul (shame not 6 months).

\end{colsection}

% ~~~~~~~~~~~~~~~~~~~~

\end{colsection}

% ########################################
