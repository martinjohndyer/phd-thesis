\chapter{Introduction}
\label{chap:intro}
\chaptoc{}

% ####################
\newpage
\section{Gravitational Waves}
\label{sec:gw}

\glspl{gw} are waves in space time.

\lipsum{}


% ~~~~~~~~~~~~~~~~~~~~
\subsection{Sources of gravitational waves}
\label{sec:gw-sources}

\lipsum{}


% ~~~~~~~~~~~~~~~~~~~~
\subsection{Detecting gravitational waves}
\label{sec:gw-detect}

\lipsum{}

% ~~~~~~~~~~~~~~~~~~~~
\subsection{Multi-messenger astronomy}
\label{sec:gw-multimessenger}

In 2016 the first direct detection of gravitational waves was announced by the LIGO collaboration \cite{PhysRevLett.116.061102}, and despite a global follow-up search no counterpart electromagnetic transient was detected \cite{2016ApJ...826L..13A}. This was not unexpected due to the signal's predicted origin as a binary black hole system. It was not until nearly two years and four more confirmed detections later that the first gravitational waves were detected from a merging binary neutron star \cite{PhysRevLett.119.161101}. This GW170817 LIGO/Virgo detection marked a milestone in the era of multi-messenger astronomy, as it was also seen in gamma-rays as GRB 170817A by the \textit{Fermi} satellite and 11 hours later as optical transient AT 2017gfo \cite{2017ApJ...848L..12A}. GW170817 was localised to a 90\% confidence interval covering 31 square degrees \cite{2017ApJ...848L..12A}, lower than the typical expected areas of hundreds of square degrees \cite{0004-637X-795-2-105}, which made prompt discovery of the associated transient possible for telescopes with a small field of view. Future detections may not be as well localised and therefore in order to ensure future counterpart observations there will be a need for wide-field searches.


% ####################
\newpage
\section{The Gravitational-wave Optical Transient Observer}
\label{sec:goto}

\lipsum{}

% ~~~~~~~~~~~~~~~~~~~~
\subsection{Motivation}
\label{sec:gw-motivation}

\lipsum{}

% ~~~~~~~~~~~~~~~~~~~~
\subsection{Design}
\label{sec:gw-design}

\lipsum{}
